\documentclass[12pt]{article}

% pacchetti e definizioni
\usepackage[utf8]{inputenc}
\usepackage[italian]{babel}
\usepackage{amsmath}
\usepackage{amssymb}
\usepackage{amsthm}
\usepackage{commath}
\usepackage{mathtools}
\usepackage{faktor}
\usepackage{color}
\usepackage{graphicx}
\usepackage{multirow}
\usepackage{mathrsfs}
\usepackage{array}
\usepackage{rotating}
\usepackage{multirow}
\usepackage{breqn}
\usepackage{tikz-cd}


\setlength{\parindent}{0pt}

\usepackage{hyperref}

\hypersetup{
    colorlinks,
    citecolor=black,
    %filecolor=black,
    linkcolor=black,
    urlcolor=black
}

\newcommand{\todo}[1]{ \marginpar{\textbf{TODO:} \textcolor{red}{#1}} }

\newtheorem{theorem}{Teorema}[section]
\newtheorem*{theorem*}{Teorema}
\newtheorem{acknowledgement}[theorem]{Acknowledgement}
\newtheorem{congettura}[theorem]{Congettura}
%\newtheorem{algorithm}[theorem]{Algoritmo}
\newtheorem{axiom}[theorem]{Assioma}
\newtheorem{case}[theorem]{Caso}
\newtheorem{claim}[theorem]{Claim}
\newtheorem{conclusion}[theorem]{Conclusione}
%\newtheorem{condition}[theorem]{Condizione}
\newtheorem{conjecture}[theorem]{Congettura}
\newtheorem{corollary}[theorem]{Corollario}
\newtheorem{criterion}[theorem]{Criterio}
\newtheorem{lemma}[theorem]{Lemma}
\newtheorem{notation}[theorem]{Notazione}
\newtheorem{problem}[theorem]{Problema}
\newtheorem{proposition}[theorem]{Proposizione}
\newtheorem{summary}[theorem]{Riassunto}

\theoremstyle{definition}
\newtheorem{definition}[theorem]{Definizione}
\newtheorem{example}[theorem]{Esempio}
\newtheorem{exercise}{Esercizio}[section]
\newtheorem{solution}[exercise]{Soluzione}
\newtheorem*{notazione}{Notazione}


\theoremstyle{remark}
\newtheorem*{oss}{Osservazione}




\DeclareMathOperator{\lcm}{lcm}
\DeclareMathOperator{\ann}{Ann}
\DeclareMathOperator{\lt}{lt}
\DeclareMathOperator{\Lt}{Lt}
\DeclareMathOperator{\Deg}{Deg}
\DeclareMathOperator{\Syl}{Syl}
\DeclareMathOperator{\Ris}{Ris}
\DeclareMathOperator{\Imm}{Im}
\DeclareMathOperator{\Dom}{Dom}
\DeclareMathOperator{\car}{char}
\DeclareMathOperator{\p}{\mathcal P}
\DeclareMathOperator{\st}{\; |\;}
\DeclareMathOperator{\et}{\;\wedge\;}
\DeclareMathOperator{\tr}{Tr}
\DeclareMathOperator{\n}{N}
\DeclareMathOperator{\disc}{disc}
\DeclareMathOperator{\GL}{GL}
\DeclareMathOperator{\SL}{SL}
\DeclareMathOperator{\Spec}{Spec}
\DeclareMathOperator{\Hom}{Hom}
\DeclareMathOperator{\End}{End}
\DeclareMathOperator{\Aut}{Aut}
\DeclareMathOperator{\h}{ht}
\DeclareMathOperator{\coh}{coht}
\DeclareMathOperator{\id}{id}
\DeclareMathOperator{\stab}{stab}
\DeclareMathOperator{\coker}{coker}
\DeclareMathOperator{\Com}{Com}
\DeclareMathOperator{\Kom}{Kom}
\DeclareMathOperator{\Ext}{Ext}
\DeclareMathOperator{\Tor}{Tor}
\DeclareMathOperator{\coind}{coInd}
\DeclareMathOperator{\ind}{Ind}
\DeclareMathOperator{\cores}{coRes}
\DeclareMathOperator{\res}{Res}
\DeclareMathOperator{\Mat}{Mat}

\newcommand{\m}{\mathfrak{m}}
\newcommand{\N}{\mathbb{N}}
\newcommand{\Z}{\mathbb{Z}}
\newcommand{\Q}{\mathbb{Q}}
\newcommand{\R}{\mathbb{R}}
\newcommand{\C}{\mathbb{C}}
\newcommand{\D}{\mathcal{D}}
\newcommand{\Hbb}{\mathbb{H}}
\newcommand{\A}{\mathbb{A}}
\newcommand{\F}{\mathbb{F}}
\newcommand{\Zn}[1]{\Z/#1\Z}
\newcommand{\gen}[1]{\ensuremath{\left< #1\right>}}
\newcommand{\normal}{\mathrel{\unlhd}}
\newcommand{\sing}[1]{\{#1\}}
\newcommand{\ds}{\displaystyle}
\newcommand{\eps}{\varepsilon}
\newcommand{\de}{\partial}

\renewcommand{\norm}[1]{\left\lVert#1\right\rVert}
\newcommand{\vp}[1]{\upsilon_p\left(#1\right)}
\newcommand{\np}[1]{\left|#1\right|_p}
\newcommand{\pf}{\mathfrak{p}}
\newcommand{\qf}{\mathfrak{q}}
\newcommand{\Pf}{\mathfrak{P}}
\newcommand{\Oc}{\mathcal{O}}
\newcommand{\vpf}[1]{\upsilon_\pf\left(#1\right)}
\newcommand{\vPf}[1]{\upsilon_\Pf\left(#1\right)}
\newcommand{\npf}[1]{\left|#1\right|_\pf}
\newcommand{\rest}[2]{\left. #1 \right|_{#2}}

\newcommand{\Gal}[2]{\operatorname{Gal}\left(\faktor{#1}{#2}\right)}


\newenvironment{esercizio}[3]{%
    \href{#2}{\textbf{#1}} (#3).
}{}





\author{Riccardo Zanotto}
\title{N1 - numeri primi}

\begin{document}

\maketitle


\section{Lezione}

\subsection{Richiami}
\begin{itemize}
    \item ordine moltiplicativo modulo $m$
    \item $a^p\equiv a\pmod p$
    \item $(p-1)!\equiv-1\pmod p$
\end{itemize}

\subsection{Cose quadratiche}

\begin{itemize}
    \item Richiami sui residui quadratici modulo cose piccole
    \item $\left( \frac{-1}{p} \right)$
    \item Equazioni di secondo grado modulo $p$
\end{itemize}

\begin{esercizio}{Lezione vecchia}{}{}
    $n^2+5n+16=169m$
\end{esercizio}

\subsection{Potenze superiori}
\begin{itemize}
    \item esistenza di generatori (solo enunciato)
    \item numero di residui $k$-esimi
    \item polinomi modulo $p$: somma su tutti i residui, principio di identità $x^{p-1}-1=(x-1)\cdots(x-(p-1))$
\end{itemize}

\begin{esercizio}{S18TF 7}{http://olimpiadi.dm.unibo.it/videolezioni/Training/Esercizi_Stage/Senior_18.pdf\#TF-N1-T}
    Qual è la cardinalità dell'immagine di $n\mapsto n^{120}\pmod{2019}$?
\end{esercizio}

\begin{esercizio}{S16TF 8}{http://olimpiadi.dm.unibo.it/videolezioni/Training/Esercizi_Stage/Senior_16.pdf}
    Calcolare $\sum_{k=0}^{1000} k^{2016}\pmod{11}$.
\end{esercizio}


\subsection{Fattori primi}

\begin{itemize}
    \item valutazione $p$-adica
    \item numero di divisori
    \item modulo $p^2$
\end{itemize}

\begin{esercizio}{S17TF 8}{http://olimpiadi.dm.unibo.it/videolezioni/Training/Esercizi_Stage/Senior_17.pdf}
    Qual è il resto di $6^{2019}+8^{2019}$ nella divisione per $49$?
\end{esercizio}

\begin{esercizio}{IMOSL18 1}{https://artofproblemsolving.com/community/c6h1876768p12752840}
    Trovare le coppie di interi $(m,n)$ per cui esiste un intero $s$ con $\sigma_0(sn)=\sigma_0(sm)$.
\end{esercizio}


\section{Esercizi alla lavagna}

\begin{esercizio}{PI18 N5}{http://olimpiadi.dm.unibo.it/videolezioni/Training/Esercizi_Stage/PreIMO_18.pdf\#BMN5.1}
    Sia $p$ un primo dispari. Dimostrare che $\left\lfloor (\sqrt5+2)^p-2^{p+1} \right\rfloor$ è multiplo di $20p$.
\end{esercizio}

\begin{esercizio}{BMO18 4}{https://artofproblemsolving.com/community/c6h1640637p10335968}
    Determinare tutte le coppie $(p,q)$ di numeri primi tali che $3p^{q-1}+1\mid11^p+17^p$.
\end{esercizio}

\begin{esercizio}{EGMO12 5}{https://artofproblemsolving.com/community/c6h474834p2659385}
    Trovare i possibili valori di $p-q$ sapendo che $\frac{p}{p+1}+\frac{q+1}{q}=\frac{2n}{n+2}$
\end{esercizio}

\section{Esercizi singoli}



\end{document}
