\documentclass[12pt]{article}

% pacchetti e definizioni
\input{../environment}




\author{Riccardo Zanotto}
\title{N1 - numeri primi}

\begin{document}

\maketitle


\section{Lezione}

\subsection{Richiami}
\begin{itemize}
    \item ordine moltiplicativo modulo $m$
    \item $a^p\equiv a\pmod p$
    \item $(p-1)!\equiv-1\pmod p$
\end{itemize}

\subsection{Cose quadratiche}

\begin{itemize}
    \item Richiami sui residui quadratici modulo cose piccole
    \item $\left( \frac{-1}{p} \right)$
    \item Equazioni di secondo grado modulo $p$
\end{itemize}

\begin{esercizio}{Lezione vecchia}{}{}
    $n^2+5n+16=169m$
\end{esercizio}

\subsection{Potenze superiori}
\begin{itemize}
    \item esistenza di generatori (solo enunciato)
    \item numero di residui $k$-esimi
    \item polinomi modulo $p$: somma su tutti i residui, principio di identità $x^{p-1}-1=(x-1)\cdots(x-(p-1))$
\end{itemize}

\begin{esercizio}{S18TF 7}{http://olimpiadi.dm.unibo.it/videolezioni/Training/Esercizi_Stage/Senior_18.pdf\#TF-N1-T}
    Qual è la cardinalità dell'immagine di $n\mapsto n^{120}\pmod{2019}$?
\end{esercizio}

\begin{esercizio}{S16TF 8}{http://olimpiadi.dm.unibo.it/videolezioni/Training/Esercizi_Stage/Senior_16.pdf}
    Calcolare $\sum_{k=0}^{1000} k^{2016}\pmod{11}$.
\end{esercizio}


\subsection{Fattori primi}

\begin{itemize}
    \item valutazione $p$-adica
    \item numero di divisori
    \item modulo $p^2$
\end{itemize}

\begin{esercizio}{S17TF 8}{http://olimpiadi.dm.unibo.it/videolezioni/Training/Esercizi_Stage/Senior_17.pdf}
    Qual è il resto di $6^{2019}+8^{2019}$ nella divisione per $49$?
\end{esercizio}

\begin{esercizio}{IMOSL18 1}{https://artofproblemsolving.com/community/c6h1876768p12752840}
    Trovare le coppie di interi $(m,n)$ per cui esiste un intero $s$ con $\sigma_0(sn)=\sigma_0(sm)$.
\end{esercizio}


\section{Esercizi alla lavagna}

\begin{esercizio}{PI18 N5}{http://olimpiadi.dm.unibo.it/videolezioni/Training/Esercizi_Stage/PreIMO_18.pdf\#BMN5.1}
    Sia $p$ un primo dispari. Dimostrare che $\left\lfloor (\sqrt5+2)^p-2^{p+1} \right\rfloor$ è multiplo di $20p$.
\end{esercizio}

\begin{esercizio}{BMO18 4}{https://artofproblemsolving.com/community/c6h1640637p10335968}
    Determinare tutte le coppie $(p,q)$ di numeri primi tali che $3p^{q-1}+1\mid11^p+17^p$.
\end{esercizio}

\begin{esercizio}{EGMO12 5}{https://artofproblemsolving.com/community/c6h474834p2659385}
    Trovare i possibili valori di $p-q$ sapendo che $\frac{p}{p+1}+\frac{q+1}{q}=\frac{2n}{n+2}$
\end{esercizio}

\section{Esercizi singoli}



\end{document}
