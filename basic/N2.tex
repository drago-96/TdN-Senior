\documentclass[12pt]{article}

% pacchetti e definizioni
\input{../environment}




\author{Riccardo Zanotto}
\title{N2 - diofantee}

\begin{document}

\maketitle


\section{Lezione}

\subsection{Fattorizzazioni e gcd}
\begin{itemize}
    \item fattorizzazioni standard
    \item interezza di una funzione razionale (di primo o secondo grado)
    \item conticini di gcd di polinomi
\end{itemize}

\begin{esercizio}{S18TF 8}{http://olimpiadi.dm.unibo.it/videolezioni/Training/Esercizi_Stage/Senior_18.pdf\#TF-N2-T}
    Determinare tutti i primi $p$ per cui esistono $a,b$ interi tali che $p^a+36=b^2$.
\end{esercizio}

\begin{esercizio}{S16TF 7}{http://olimpiadi.dm.unibo.it/videolezioni/Training/Esercizi_Stage/Senior_16.pdf}
    Determinare il massimo $n$ naturale per cui $n+3\mid n^4+2016$.
\end{esercizio}

\begin{esercizio}{EGMO13 4}{https://artofproblemsolving.com/community/c6h529188p3014762}
    Trovare tutti gli interi $a,b$ per cui esistono tre interi consecutivi sui quali il polinomio $P(n)=\frac{n^5+a}{b}$ assume valori interi.
\end{esercizio}


\subsection{Disuguaglianze}
\begin{itemize}
    \item gli interi saltano di 1!
    \item stringere tra due quadrati
    \item ordini di grandezza di polinomi/esponenziali/fattoriali
\end{itemize}

\begin{esercizio}{Lezione vecchia}{}
    Trovare tutti gli $a,b,c$ interi per cui si ha $a^n+b^n+c^n=0$ per infiniti interi $n$.
\end{esercizio}

\begin{esercizio}{S18TI 14}{http://olimpiadi.dm.unibo.it/videolezioni/Training/Esercizi_Stage/Senior_18.pdf\#TI-N2-T}
    Trovare tutte le soluzioni intere di $n^2-6n=m^2+m-12$.
\end{esercizio}

\begin{esercizio}{EGMO15 4}{https://artofproblemsolving.com/community/c6h1078897p4728593}
    Determinare se esiste una successione di interi positivi $a_1,a_2,\dots$ tali che $a_{n+2}=a_{n+1}+\sqrt{a_{n+1}+a_n}$.
\end{esercizio}

\begin{esercizio}{BMO17 1}{https://artofproblemsolving.com/community/c6h1441692p8209533}
    $x^3+y^3=x^2+42xy+y^2$ nei positivi.
\end{esercizio}


\subsection{Congruenze}
\begin{itemize}
    \item Residui modulo cose piccole
    \item Trovare esponenti pari
    \item Guardare modulo potenze di primi
\end{itemize}

\begin{esercizio}{S17TF 12}{http://olimpiadi.dm.unibo.it/videolezioni/Training/Esercizi_Stage/Senior_17.pdf}
    Determinare tutte le quaterne $(x,y,z,w)$ di interi non negativi tali che $2^x+3^y+5^z=7^w$.
\end{esercizio}

Qualche diofantea con i fattoriali

\subsection{Altre tecniche}
\begin{itemize}
    \item discesa infinita
    \item delta=quadrato
\end{itemize}

\begin{esercizio}{Lezione vecchia}{}
    Per quali primi $p$ il polinomio $x^2+px-444p$ ha radici intere?
\end{esercizio}

\section{Esercizi alla lavagna}

Chicken McNuggets theorem

$7=q_1^2+q_2^2+q_3^2$.


\begin{esercizio}{S17TI 13}{http://olimpiadi.dm.unibo.it/videolezioni/Training/Esercizi_Stage/Senior_17.pdf}
    Quanti sono i valori di $n$ per cui $n^2+85n+2017$ è un quadrato perfetto?
\end{esercizio}

\begin{esercizio}{BMO17 3}{https://artofproblemsolving.com/community/c6h1441693p8209540}
    Trovare tutte le funzioni $f:\N\to\N$ tali che $n+f(m)\mid f(n)+nf(m)$.
\end{esercizio}

\begin{esercizio}{IMO16 4}{}
    Numeri profumati
\end{esercizio}



\section{Esercizi singoli}

\begin{esercizio}{S18TF 12}{http://olimpiadi.dm.unibo.it/videolezioni/Training/Esercizi_Stage/Senior_18.pdf\#TF-ND-S}
    Determinare tutte le coppie $(a,b)$ di interi positivi tali che $9^a-7^a=2^b$.
\end{esercizio}

\begin{esercizio}{}{}
    $mn+2m-n-8=0$
\end{esercizio}

\begin{esercizio}{}{}
    $5p+49=a^2$
\end{esercizio}

\begin{esercizio}{}{}
    $3^x-y^2=41$
\end{esercizio}





\end{document}
