\documentclass[12pt]{article}

% pacchetti e definizioni
\input{../environment}




\author{Riccardo Zanotto}
\title{N2 - mistone}

\begin{document}

\maketitle


\section{Lezione}

Programma indicativo:

\begin{itemize}
    \item approssimazione diofantea e Pell
    \item reciprocità quadratica
    \item somme di simboli di Legendre
    \item Vieta jumping
    \item \textit{funzioni aritmetiche e loro convoluzioni}
\end{itemize}

\section{Esercizi alla lavagna}

\begin{esercizio}{S17TI 16}{http://olimpiadi.dm.unibo.it/videolezioni/Training/Esercizi_Stage/Senior_17.pdf}{**}
    Determinare tutti i possibili valori di $c$ per cui esista almeno una soluzione di $a^2+b^2-abc+1=0$
\end{esercizio}

\begin{esercizio}{S16TI 16}{http://olimpiadi.dm.unibo.it/videolezioni/Training/Esercizi_Stage/Senior_17.pdf}{**}
    Determinare per quanti valori interi di $a$ il numero $2a^2+27+91$ è un quadrato perfetto.
\end{esercizio}

\section{Esercizi singoli}



\end{document}
