\documentclass[12pt]{article}

% pacchetti e definizioni
\usepackage[utf8]{inputenc}
\usepackage[italian]{babel}
\usepackage{amsmath}
\usepackage{amssymb}
\usepackage{amsthm}
\usepackage{commath}
\usepackage{mathtools}
\usepackage{faktor}
\usepackage{color}
\usepackage{graphicx}
\usepackage{multirow}
\usepackage{mathrsfs}
\usepackage{array}
\usepackage{rotating}
\usepackage{multirow}
\usepackage{breqn}
\usepackage{tikz-cd}


\setlength{\parindent}{0pt}

\usepackage{hyperref}

\hypersetup{
    colorlinks,
    citecolor=black,
    %filecolor=black,
    linkcolor=black,
    urlcolor=black
}

\newcommand{\todo}[1]{ \marginpar{\textbf{TODO:} \textcolor{red}{#1}} }

\newtheorem{theorem}{Teorema}[section]
\newtheorem*{theorem*}{Teorema}
\newtheorem{acknowledgement}[theorem]{Acknowledgement}
\newtheorem{congettura}[theorem]{Congettura}
%\newtheorem{algorithm}[theorem]{Algoritmo}
\newtheorem{axiom}[theorem]{Assioma}
\newtheorem{case}[theorem]{Caso}
\newtheorem{claim}[theorem]{Claim}
\newtheorem{conclusion}[theorem]{Conclusione}
%\newtheorem{condition}[theorem]{Condizione}
\newtheorem{conjecture}[theorem]{Congettura}
\newtheorem{corollary}[theorem]{Corollario}
\newtheorem{criterion}[theorem]{Criterio}
\newtheorem{lemma}[theorem]{Lemma}
\newtheorem{notation}[theorem]{Notazione}
\newtheorem{problem}[theorem]{Problema}
\newtheorem{proposition}[theorem]{Proposizione}
\newtheorem{summary}[theorem]{Riassunto}

\theoremstyle{definition}
\newtheorem{definition}[theorem]{Definizione}
\newtheorem{example}[theorem]{Esempio}
\newtheorem{exercise}{Esercizio}[section]
\newtheorem{solution}[exercise]{Soluzione}
\newtheorem*{notazione}{Notazione}


\theoremstyle{remark}
\newtheorem*{oss}{Osservazione}




\DeclareMathOperator{\lcm}{lcm}
\DeclareMathOperator{\ann}{Ann}
\DeclareMathOperator{\lt}{lt}
\DeclareMathOperator{\Lt}{Lt}
\DeclareMathOperator{\Deg}{Deg}
\DeclareMathOperator{\Syl}{Syl}
\DeclareMathOperator{\Ris}{Ris}
\DeclareMathOperator{\Imm}{Im}
\DeclareMathOperator{\Dom}{Dom}
\DeclareMathOperator{\car}{char}
\DeclareMathOperator{\p}{\mathcal P}
\DeclareMathOperator{\st}{\; |\;}
\DeclareMathOperator{\et}{\;\wedge\;}
\DeclareMathOperator{\tr}{Tr}
\DeclareMathOperator{\n}{N}
\DeclareMathOperator{\disc}{disc}
\DeclareMathOperator{\GL}{GL}
\DeclareMathOperator{\SL}{SL}
\DeclareMathOperator{\Spec}{Spec}
\DeclareMathOperator{\Hom}{Hom}
\DeclareMathOperator{\End}{End}
\DeclareMathOperator{\Aut}{Aut}
\DeclareMathOperator{\h}{ht}
\DeclareMathOperator{\coh}{coht}
\DeclareMathOperator{\id}{id}
\DeclareMathOperator{\stab}{stab}
\DeclareMathOperator{\coker}{coker}
\DeclareMathOperator{\Com}{Com}
\DeclareMathOperator{\Kom}{Kom}
\DeclareMathOperator{\Ext}{Ext}
\DeclareMathOperator{\Tor}{Tor}
\DeclareMathOperator{\coind}{coInd}
\DeclareMathOperator{\ind}{Ind}
\DeclareMathOperator{\cores}{coRes}
\DeclareMathOperator{\res}{Res}
\DeclareMathOperator{\Mat}{Mat}

\newcommand{\m}{\mathfrak{m}}
\newcommand{\N}{\mathbb{N}}
\newcommand{\Z}{\mathbb{Z}}
\newcommand{\Q}{\mathbb{Q}}
\newcommand{\R}{\mathbb{R}}
\newcommand{\C}{\mathbb{C}}
\newcommand{\D}{\mathcal{D}}
\newcommand{\Hbb}{\mathbb{H}}
\newcommand{\A}{\mathbb{A}}
\newcommand{\F}{\mathbb{F}}
\newcommand{\Zn}[1]{\Z/#1\Z}
\newcommand{\gen}[1]{\ensuremath{\left< #1\right>}}
\newcommand{\normal}{\mathrel{\unlhd}}
\newcommand{\sing}[1]{\{#1\}}
\newcommand{\ds}{\displaystyle}
\newcommand{\eps}{\varepsilon}
\newcommand{\de}{\partial}

\renewcommand{\norm}[1]{\left\lVert#1\right\rVert}
\newcommand{\vp}[1]{\upsilon_p\left(#1\right)}
\newcommand{\np}[1]{\left|#1\right|_p}
\newcommand{\pf}{\mathfrak{p}}
\newcommand{\qf}{\mathfrak{q}}
\newcommand{\Pf}{\mathfrak{P}}
\newcommand{\Oc}{\mathcal{O}}
\newcommand{\vpf}[1]{\upsilon_\pf\left(#1\right)}
\newcommand{\vPf}[1]{\upsilon_\Pf\left(#1\right)}
\newcommand{\npf}[1]{\left|#1\right|_\pf}
\newcommand{\rest}[2]{\left. #1 \right|_{#2}}

\newcommand{\Gal}[2]{\operatorname{Gal}\left(\faktor{#1}{#2}\right)}


\newenvironment{esercizio}[3]{%
    \href{#2}{\textbf{#1}} (#3).
}{}





\author{Riccardo Zanotto}
\title{Approssimazione diofantea}

\begin{document}

\maketitle

\section{Lezione}

\subsection{Dirichlet} 

\begin{theorem}
    Siano $x_1,\dots,x_k$ reali e $\eps>0$. Esiste un intero $n$ e interi $p_1,\dots,p_k$ tali che $|nx_i-p_i|<\eps$.
\end{theorem}

\begin{esercizio}{PFTB 15.1.6}{}
    Siano $x_1,x_2,\dots,x_{2n+1}$ numeri reali tali che per ogni $1\le i\le2n+1$ posso dividere in due gruppi gli $x_j$ con $j\neq i$ in modo che la somma dei due gruppi sia uguale. Dimostrare che tutti i numeri sono uguali.
\end{esercizio}

\subsection{Cose su $\{ n\alpha \}$}

\begin{lemma}
     Le potenze iniziano con cifre arbitrarie
\end{lemma}

\begin{esercizio}{PFTB 15.1.1}{}
    Mostrare che la successione $\lfloor n\sqrt{2019}\rfloor$ contiene progressioni geometriche arbitrariamente lunghe con ragioni arbitrariamente grandi.
\end{esercizio}

\subsection{Teorema di Weyl}

\begin{theorem}
    Sia $a_n$ una successione in $[0,1]$. Le seguenti sono equivalenti:
    \begin{enumerate}
        \item Per ogni $0\le a\le b\le 1$ vale $$ \lim_{n\to\infty}\frac{\#\{ i\st 1\le i\le n, a_i\in[a,b] \}}{n} = b-a $$
        \item Per ogni $f:[0,1]\to\R$ continua vale $$ \lim_{n\to\infty}\frac{1}{n}\sum_{i=1}^n f(a_i)=\int_0^1 f(x)\mathrm{d}x$$
        \item Per ogni intero $r\ge1$ vale $$ \lim_{n\to\infty}\frac1n\sum_{k=1}^ne^{2i\pi r a_k} = 0 $$
    \end{enumerate}
    Se vale una di queste, la successione è \emph{equidistribuita}.
\end{theorem}

\begin{corollary}
    La successione $\{ n\alpha \}$ è equidistribuita per $\alpha$ irrazionale.
\end{corollary}

\begin{corollary}
    La densità degli interi $n$ per cui $a^n$ comincia con le cifre $M$ è $\ds\frac{\log(M+1)}{\log M}$.
\end{corollary}

\begin{theorem}
    Se $f$ è un polinomio con coefficiente di testa irrazionale, allora la successione $f(n)$ è equidistribuita.
\end{theorem}

\begin{esercizio}{PFTB 15.2.5}{}
    Nella successione $\lfloor n^2\sqrt{2019} \rfloor$ ci sono progressioni geometriche arbitrariamente lunghe.
\end{esercizio}

\begin{exercise}
    Dimostare che esiste un $\alpha$ irrazionale tale che $\{ \alpha^n \}$ non è equidistibuita. E $\alpha$ trascendente?
\end{exercise}

\subsection{Beatty}
\begin{theorem}
    Siano $r,s$ due irrazionali positivi tali che $\frac1r+\frac1s=1$. Allora le due successioni $\lfloor nr\rfloor$ e $\lfloor ns \rfloor$ partizionano $\N$.
\end{theorem}

\begin{esercizio}{PFTB 15.2.12}{}
    Siano $a,b,c$ reali positivi. Mostrare che $\N$ non può essere partizionato nelle successioni $\lfloor na\rfloor$, $\lfloor nb\rfloor$, $\lfloor nc\rfloor$.
\end{esercizio}





\newpage

\section{Esercizi singoli}

\begin{esercizio}{Engel}{}
    Sia $f(n)=\lfloor n+\sqrt n+\frac12 \rfloor$. Mostrare che $f(n)$ manca esattamente i quadrati.
\end{esercizio}

\begin{esercizio}{AOPS}{https://artofproblemsolving.com/community/c6h1661563p10549146}
    Siano $\alpha,\beta$ irrazionali tali che $\lfloor \alpha\lfloor \beta x \rfloor \rfloor = \lfloor \beta\lfloor \alpha x \rfloor \rfloor$ per ogni $x>0$. Dimostrare che $\alpha=\beta$.
\end{esercizio}

\begin{esercizio}{AOPS}{https://artofproblemsolving.com/community/c6h1634940p10278481}
    Siano $a_1\le\dots\le a_{25}$ interi non negativi. Mostrare che $\lfloor \sqrt{a_1} \rfloor+\dots+\lfloor \sqrt{a_1} \rfloor\ge\left\lfloor \sqrt{a_1+\dots+a_{25}+200a_1} \right\rfloor$
\end{esercizio}

\begin{esercizio}{IMOSL15 N1}{https://artofproblemsolving.com/community/c6h1268859p6622268}
    Determinare tutti gli $M$ tali che la successione $a_0=M+\frac1 2$, $a_{k+1}=a_k\lfloor a_k\rfloor$ contenga almeno un intero.
\end{esercizio}

\begin{esercizio}{BMO15 4}{https://artofproblemsolving.com/community/c6h1085437p4794940}
    Dimostrare che per ogni $20$ interi consecutivi essite un $d$ tale che per ogni $n$ naturale vale la disuguaglianza $n\sqrt d\left\{ n\sqrt d \right\}>\frac 5 2$.
\end{esercizio}

\begin{esercizio}{PFTB 15.2.3}{}
    Calcolare $\ds\sup_{n\ge1}\left( \min_{p+q=n} |p-q\sqrt3| \right)$
\end{esercizio}

\begin{esercizio}{PFTB 15.2.7}{}
    Dimostrare che definitivamente per ogni $n$ esistono degli interi positivi $a,b$ tali che $\lfloor a\sqrt2+b\sqrt{2019} \rfloor=n$.
\end{esercizio}

\begin{esercizio}{PFTB 15.2.11}{}
    Siano $a,b$ reali tali che $\{na\}+\{nb\}<1$ per ogni $n$. Dimostrare che almeno uno tra $a$ e $b$ è intero.
\end{esercizio}

\begin{esercizio}{IMOSL14 N8}{https://artofproblemsolving.com/community/c6h1113204p5083578}
    Dimostrare che per ogni $a,b$ interi esiste un $p^k$ dispari tale che $\norm{\frac{a}{p^k}}+\norm{\frac{b}{p^k}}+\norm{\frac{a+b}{p^k}}=1$, dove $\norm x$ è l'intero più vicino a $x$.
\end{esercizio}

\begin{esercizio}{RMM15 5}{https://artofproblemsolving.com/community/c6h1058252p4575509}
    Sia $p\ge5$ un numero primo; definiamo $R(k)$ il resto di $k$ nella divisione per $p$. Determinare tutti gli interi $a<p$ tali che $m+R(ma)>a$ per ogni $m=1,\dots,p-1$.
\end{esercizio}



\end{document}